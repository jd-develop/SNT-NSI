\documentclass[11pt]{article} % Font size
\usepackage{amsmath, amsfonts, amssymb, amsthm, mathrsfs}
\usepackage[french]{babel}
\usepackage[french]{isodate}
\usepackage{zref-totpages}
\usepackage{pgf}
\usepackage{listings}
\usepackage{hyperref}

% \usepackage{siunitx}
%\usepackage{array}
%\usepackage{graphicx}
%\usepackage{subcaption}
%\usepackage{float}
% \usepackage{tikz}
% \usetikzlibrary{babel}
% \usetikzlibrary{trees}
% \usetikzlibrary{arrows}
% \usepackage{circuitikz}
% \tikzset{every picture/.style={line width=0.75pt}} %set default line width to 0.75pt

\usepackage[utf8]{inputenc} % Required for inputting international characters
\usepackage[T1]{fontenc} % Use 8-bit encoding

\usepackage{geometry} % Required for adjusting page dimensions and margins
\geometry{
    paper=a4paper, % Paper size, change to letterpaper for US letter size
    top=2cm, % Top margin
    bottom=2.5cm, % Bottom margin
    left=2cm, % Left margin
    right=2cm, % Right margin
    headheight=0.75cm, % Header height
    footskip=1cm, % Space from the bottom margin to the baseline of the footer
    headsep=0.5cm, % Space from the top margin to the baseline of the header
    %showframe, % Uncomment to show how the type block is set on the page
}

\definecolor{codegreen}{rgb}{0,0.6,0}
\definecolor{codegray}{rgb}{0.5,0.5,0.5}
\definecolor{codepurple}{rgb}{0.58,0,0.82}
\definecolor{backcolour}{rgb}{0.95,0.95,0.92}

\lstdefinestyle{mystyle}{
    backgroundcolor=\color{backcolour},
    commentstyle=\color{codegreen},
    keywordstyle=\color{magenta},
    numberstyle=\tiny\color{codegray},
    stringstyle=\color{codepurple},
    basicstyle=\ttfamily\footnotesize,
    breakatwhitespace=false,
    breaklines=true,
    captionpos=b,
    keepspaces=true,
    numbers=left,
    numbersep=5pt,
    showspaces=false,
    showstringspaces=false,
    showtabs=false,
    tabsize=4
}

\lstset{
    style=mystyle,
    inputencoding = utf8  % Input encoding
}

\hypersetup{colorlinks=true, urlcolor=blue}

\renewcommand{\thesection}{\Roman{section}}
\renewcommand{\thesubsubsection}{\alph{subsubsection})}
\newcommand{\ol}{\overline}
\newcommand{\nbsp}{\nobreakspace}

%------------------------------------------------------------------------------
%    TITLE SECTION
%------------------------------------------------------------------------------

\title{
    \vspace{10pt} % Whitespace
    \rule{\linewidth}{1pt}\\ % Thin top horizontal rule
    \vspace{10pt} % Whitespace
    {\huge Rapport de DM}\\ % The assignment title
    % \vspace{1pt} % Whitespace
    \rule{\linewidth}{1pt}\\ % Thick bottom horizontal rule
    \vspace{5pt} % Whitespace
}

\date{\printdate{2024-12-29}}
\author{TODO}

\begin{document}
\maketitle{}

\section{Questions}
\paragraph{Q1.}
La valeur de l’échantillon numéro $i$ du son sinusoïdal de fréquence $f$ et
d’amplitude $A$ est\nbsp:
\begin{align*}
A \sin(2\pi f i \tau_\text{ech})
\end{align*}

\paragraph{Q3.}
\nbsp

\begin{lstlisting}
000000 52 49 46 46 2e 00 00 00 57 41 56 45 66 6d 74 20
000010 10 00 00 00 01 00 01 00 22 56 00 00 44 ac 00 00
000020 02 00 10 00 64 61 74 61 0a 00 00 00 d2 03 5e 06
000030 ff ff a2 f9 ff f7
\end{lstlisting}

\paragraph{Q18.}
La complexité de la fonction \verb|reduce_mix| est de l’ordre de
$\mathcal O(n+n\max(l_1, \dots, l_n))$ car elle contient trois boucles\nbsp:
deux en $\mathcal O(n)$ et une en $\mathcal O(n\max(l_1, \dots, l_n))$.

\section{Description}

\subsection{Description des fichiers}
\begin{enumerate}
    \item \verb|src/run_tests.c|\nbsp: la fonction exécutant tous les tests
        du programme. Cette fonction est appelée par \verb|main.c|.
        Note au passage\nbsp: ce programme ne peut tourner que sur des machines
        POSIX (Unix ou GNU\footnote{Indépendamment du noyau (Linux ou Hurd)}),
        à cause du fait que \verb|src/run_tests.c| écrit des fichiers dans
        \verb|/tmp|.
    \item \verb|src/sound.c|\nbsp: fichier lié à la gestion des sons
        (\verb|sound_t|)
    \item \verb|src/wav.c|\nbsp: fichier chargé d’écrire les fichiers WAV.
        Contient aussi la fonction \verb|write_int|.
\end{enumerate}

\end{document}
