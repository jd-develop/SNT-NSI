\documentclass[a4paper, 10pt]{article}

%% Importations des paquets
\usepackage[margin=2.0cm]{geometry} % Modifie la taille des marges
\usepackage{parskip, multicol} % Modifie la disposition des paragraphes et des colonnes
\usepackage{xcolor} % Permet l'utilisation des couleurs

\usepackage[french]{babel} % Met le document en français
\usepackage[utf8]{inputenc} % Encode le fichier en Utf-8
\usepackage[T1]{fontenc} % Affiche les caractères accentués

\usepackage{amsthm, amssymb, amsfonts, mathtools} % Ajoute des outils mathématiques
\usepackage{siunitx} % Ajouts des outils physiques
\usepackage{listings, graphicx, tikz} % Permet d’importer des fichiers, images et graphiques
\usepackage[straightvoltages, RPvoltages, european, american inductor]{circuitikz} % Permet d’importer des circuits en pgf

%% Paramètres des paquets
\lstset{
    frame=single, % Paramètre pour listings
    numbers=left,
    breaklines=true,
    basicstyle=\scriptsize\ttfamily\color{black},
    keywordstyle=\bfseries\color{blue},
    commentstyle=\color{gray},
    stringstyle=\color{purple},
    showspaces=false,
    showstringspaces=false,
    showtabs=false,
    tabsize=4,
    escapeinside={§}{§},
    literate={é}{{\'e}}1
    }

%% Theoreme
\newtheorem{theorem}{Théorème}
\newtheorem{corollary}{Corollaire}[theorem]
\newtheorem{lemma}[theorem]{Lemme}

\theoremstyle{definition}
\newtheorem{exercice}{Exercice}
\newtheorem{question}{Question}
\newtheorem*{comment}{Commentaire}

%% Macro
\newcommand*{\bb}[1]{\mathbb{#1}} % Affiche L'ensemble #1
\renewcommand*{\rm}[1]{\mathrm{#1}} % Affiche #1 en Roman
\renewcommand*{\bf}[1]{\mathbf{#1}} % Affiche #1 en gras
\renewcommand*{\cal}[1]{\mathcal{#1}} % Affiche #1 en script

\newcommand*{\obar}[1]{\overline{#1}} % Affiche #1 bar
\newcommand*{\ubar}[1]{\underline{#1}} % Affiche #1 sous bar

\newcommand*{\deriv}[2]{\frac{\rm{d}#1}{\rm{d}#2}} % Affiche la derivee de #1 par #2
\newcommand*{\dderiv}[3][2]{\deriv{^{#1}#2}{#3^{#1}}} % Affiche la derivee #1-ieme de #2 par #3
\newcommand*{\pderiv}[2]{\frac{\partial#1}{\partial#2}} % Affiche la derivee partielle de #1 par #2
\newcommand*{\pdderiv}[3][2]{\derivp{^{#1}#2}{#3^{#1}}} % Affiche la derivee partielle #1-ieme de #2 par #3

\newcommand*{\txt}[2][]{\lstinputlisting[language=#1]{#2}} % Affiche le fichier #2 dans le langage #1
\newcommand*{\img}[2][8cm]{\begin{center} % Affiche l'image #2 avec une hauteur de #1
                           \includegraphics[height=#1]{#2}
                           \end{center}}
\newcommand*{\pgf}[2][8cm]{\begin{center} % Affiche le pgf #2 avec une hauteur de #1
                           \shorthandoff{:;!?}
                           \resizebox{!}{#1}{\input{#2}}
                           \end{center}}
