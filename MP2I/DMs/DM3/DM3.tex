\documentclass[a4paper, 10pt]{article}

%% Importations des paquets
\usepackage[margin=2.0cm]{geometry} % Modifie la taille des marges
\usepackage{parskip, multicol} % Modifie la disposition des paragraphes et des colonnes
\usepackage{xcolor} % Permet l'utilisation des couleurs

\usepackage[french]{babel} % Met le document en français
\usepackage[utf8]{inputenc} % Encode le fichier en Utf-8
\usepackage[T1]{fontenc} % Affiche les caractères accentués

\usepackage{amsthm, amssymb, amsfonts, mathtools} % Ajoute des outils mathématiques
\usepackage{siunitx} % Ajouts des outils physiques
\usepackage{listings, graphicx, tikz} % Permet d’importer des fichiers, images et graphiques
\usepackage[straightvoltages, RPvoltages, european, american inductor]{circuitikz} % Permet d’importer des circuits en pgf

%% Paramètres des paquets
\lstset{frame=single, % Paramètre pour listings
        numbers=left,
        breaklines=true,
        basicstyle=\scriptsize\ttfamily\color{black},
        keywordstyle=\bfseries\color{blue},
        commentstyle=\color{gray},
        stringstyle=\color{purple},
        showspaces=false,
        showstringspaces=false,
        showtabs=false,
        tabsize=4,
        escapeinside={§}{§}
        }

%% Theoreme
\newtheorem{theorem}{Théorème}
\newtheorem{corollary}{Corollaire}[theorem]
\newtheorem{lemma}[theorem]{Lemme}

\theoremstyle{definition}
\newtheorem{exercice}{Exercice}
\newtheorem{question}{Question}
\newtheorem*{comment}{Commentaire}

%% Macro
\newcommand*{\bb}[1]{\mathbb{#1}} % Affiche L'ensemble #1
\renewcommand*{\rm}[1]{\mathrm{#1}} % Affiche #1 en Roman
\renewcommand*{\bf}[1]{\mathbf{#1}} % Affiche #1 en gras
\renewcommand*{\cal}[1]{\mathcal{#1}} % Affiche #1 en script

\newcommand*{\obar}[1]{\overline{#1}} % Affiche #1 bar
\newcommand*{\ubar}[1]{\underline{#1}} % Affiche #1 sous bar

\newcommand*{\deriv}[2]{\frac{\rm{d}#1}{\rm{d}#2}} % Affiche la derivee de #1 par #2
\newcommand*{\dderiv}[3][2]{\deriv{^{#1}#2}{#3^{#1}}} % Affiche la derivee #1-ieme de #2 par #3
\newcommand*{\pderiv}[2]{\frac{\partial#1}{\partial#2}} % Affiche la derivee partielle de #1 par #2
\newcommand*{\pdderiv}[3][2]{\derivp{^{#1}#2}{#3^{#1}}} % Affiche la derivee partielle #1-ieme de #2 par #3

\newcommand*{\txt}[2][]{\lstinputlisting[language=#1]{#2}} % Affiche le fichier #2 dans le langage #1
\newcommand*{\img}[2][8cm]{\begin{center} % Affiche l'image #2 avec une hauteur de #1
                            \includegraphics[height=#1]{#2}
                            \end{center}}
\newcommand*{\pgf}[2][8cm]{\begin{center} % Affiche le pgf #2 avec une hauteur de #1
                            \shorthandoff{:;!?}
                            \resizebox{!}{#1}{\input{#2}}
                            \end{center}}
\newcommand{\nbsp}{\nobreakspace}


\title{DM3: Conception et utilisation d’un SAT-solver}
\author{Nils OGER}
\date{}

\begin{document}
	\maketitle
	\section{Conception du SAT-solver}
	\subsection*{Q10.}
	Soit $n$ le nombre d'opérateur.
	
	Dans la configuration $(...((a_0 | a_1) | a_2) | ...) | a_n$,
	il y a bien $n$ opérateur.
	
	De plus, la complexité de l'algorithme dans cette configuration est
	\begin{align*}
		C_n &= C_{n-1} + C_0 + \Theta(n)\\
		&\geq C_{n-1} + nA\\
		&\geq C_0 + \sum_{i=0}^{n-1}(n-i)A\\
		&= \Omega(n^2)
	\end{align*}
	Donc dans le pire cas, la complexité est au moins en $\Omega(n^2)$.
	
	Il faut passer par une variable intermédiaire puis trier la liste.
	
	\subsection*{Q11.(Bonus)}
	Dans la nouvelle fonction, on a
	\begin{align*}
		C_n &= \Theta(n) + \Theta(n\log(n))\\
		&= \Theta(n\log(n))
	\end{align*}
	
	\subsection*{Q19.}
	Dans la configuration $\sim(...(\sim T)...)$, 
	la complexité de l'algorithme dans cette configuration est
	\begin{align*}
		C_n &= C_{n-2} + \Theta(n)\\
		&= \Theta(n^2)
	\end{align*}
	
	\subsection*{Q20.}
	Dans la nouvelle fonction, on simplifie les enfants avant le noeud.
	
	La complexité est
	\begin{align*}
		C_n &= C_{n-1} + \Theta(1)\\
		&= \Theta(n)
	\end{align*}
	
	\subsection*{Q25.}~
	
	\section{Résolution de problèmes}
	\subsection*{Q31.}
	La formule $\bigwedge\limits_{1 \leq i < j \leq n}(\lnot a_i \vee \lnot a_j)$ est sous FNC.
	
	\subsection*{Q38.}~
	
	\subsection*{Q40.}
	Pour le problème a 8 dames, on obtient
	\begin{lstlisting}[literate={é}{{\'{e}}}{1}]
~/DM3/problemes$ ./n_dames 8
Fichier '8_dames.txt' généré
Taille du fichier: 15191 octets
~/DM3/problemes$ ../satsolver/satsolver 8_dames.txt 
La formule est satisfiable en assignant 1 aux variables suivantes et 0 aux autres:
X_0_7
X_1_3
X_2_0
X_3_2
X_4_5
X_5_1
X_6_6
X_7_4
Temps d'exécution : 0.139068 s
	\end{lstlisting}
	Soit encore
	\begin{center}
		\begin{tabular}{| c || *{8}{c |}}
		\hline
		  & 0 & 1 & 2 & 3 & 4 & 5 & 6 & 7 \\
		\hline
		\hline
		0 &   &   & X &   &   &   &   &   \\
		\hline
		1 &   &   &   &   &   & X &   &   \\
		\hline
		2 &   &   &   & X &   &   &   &   \\
		\hline
		3 &   & X &   &   &   &   &   &   \\
		\hline
		4 &   &   &   &   &   &   &   & X \\
		\hline
		5 &   &   &   &   & X &   &   &   \\
		\hline
		6 &   &   &   &   &   &   & X &   \\
		\hline
		7 & X &   &   &   &   &   &   &   \\
		\hline
		\end{tabular}
	\end{center}
	Pour le problème a 5 dames, on obtient
	\begin{center}
		\begin{tabular}{| c || *{5}{c |}}
		\hline
		  & 0 & 1 & 2 & 3 & 4 \\
		\hline
		\hline
		0 &   &   & X &   &   \\
		\hline
		1 &   &   &   &   & X \\
		\hline
		2 &   & X &   &   &   \\
		\hline
		3 &   &   &   & X &   \\
		\hline
		4 & X &   &   &   &   \\
		\hline
		\end{tabular}
	\end{center}
	Et pour le problème a 3 dames, on obtient
	\begin{lstlisting}[literate={é}{{\'{e}}}{1}]
~/DM3/problemes$ ./n_dames 3
Fichier '3_dames.txt' généré
Taille du fichier: 671 octets
~/DM3/problemes$ ../satsolver/satsolver 3_dames.txt 
La formule est insatisfiable
Temps d'exécution : 0.002545 s
	\end{lstlisting}
\end{document}
