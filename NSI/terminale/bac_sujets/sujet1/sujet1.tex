\documentclass{article}
\usepackage{geometry}

\title{Sujet NSI Amérique du Nord}
\author{Sujet 1}
\date{}

\begin{document}

\maketitle

\section*{Exercice 1}
\paragraph*{1.}
\begin{verbatim}
    def echange(tab, i, j):
        tab[i], tab[j] = tab[j], tab[i]
\end{verbatim}

\paragraph*{2.}
\begin{verbatim}
    def triStooge(tab, i, j):
        if tab[i] > tab[j]:
            echange(tab, i, j)
        if (j - i) > 1:
            k = (j - i + 1)//3
            triStooge(tab, i, j-k)
            triStooge(tab, i+k, j)
            triStooge(tab, i, j-k)
\end{verbatim}

\paragraph*{3.}
Cet algorithme est récursif\nobreakspace: en effet, la fonction \verb|triStooge| s’appelle elle-même.

\paragraph*{4.}
Lors du premier appel, on échange la première et la dernière valeur (5 et 1).
\verb|(j-i)| est supérieur à 1 ($j-i=5$), on calcule donc $k=(j-i+1)//3 = 6//3 = 2$.

\paragraph*{5.}
Il y en a 39.

\paragraph*{6.}
Case 1\nobreakspace:\verb|triStooge(A, 1, 3)|

Case 2\nobreakspace:\verb|triStooge(A, 2, 3)|

Case 3\nobreakspace:\verb|triStooge(A, 0, 3)|

\paragraph*{7.}

\paragraph*{8.}
Le tri rapide (\verb|quicksort|) a une complexité de l’ordre de $n \log(n)$, soit
strictement meilleur que $n^e$.

\section*{Exercice 2}
\paragraph*{1.}
Dufour, Marc

Martin, Sophie

\paragraph*{2.}
\begin{verbatim}
    SELECT nom_medic FROM medicament WHERE prix < 3
\end{verbatim}

\paragraph*{3.}
\begin{verbatim}
    INSERT INTO client (id_client, nom_client, prenom_client, num_secu_sociale)
    VALUES (3, "Durand", "Nathalie", "269054958815780")
\end{verbatim}

\paragraph*{4.}
Les clés \verb|id_client| et \verb|id_medic| doivent être des clés étrangères
(respectivement vers les tables \verb|client| et \verb|medicament|). Cela permet
à la fois de lier deux tables sans re-rentrer toutes les valeurs à chaque fois.
Cela permet aussi de vérifier si le client est en effet dans la base, ou si le médicament
y est.

\paragraph*{5.}
Il faut une boîte de Paracétamol 1 gramme CP de 8 comprimés
(au maximum 3 comprimés par jour pendant 2 jours\nobreakspace: $3 \times 2 = 6 < 8$)

Il faut 3 boîtes d’acide ascorbique (1 comprimé par jour pendant 4 semaines\nobreakspace:
$4\times7=28<30=3\times10$).

Ainsi dans les lignes 7 et 8, le nombre de boîtes doit être 1 et 3.

\end{document}
